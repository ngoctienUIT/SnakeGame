\documentclass[13pt,a4paper]{article}
\usepackage[T5]{fontenc}
\usepackage[utf8]{inputenc}
\usepackage{amsmath}
\usepackage[colorinlistoftodos]{todonotes}
\usepackage{listings}
\usepackage{hyperref}
\usepackage{diagbox}
\usepackage{array}
\usepackage{url}
\usepackage{multirow}
\usepackage{lineno}
\usepackage{makeidx}
\usepackage{indentfirst}
\usepackage[utf8]{vietnam}
\usepackage{graphicx}
\usepackage{niceframe}
\usepackage{array}
\usepackage{float}
\usepackage{hyperref}
\usepackage{titlesec}
\usepackage{minitoc}
\usepackage[left = 3cm, right = 2cm, top = 2cm, bottom = 2cm] {geometry}
\hypersetup{
    colorlinks=true,
    linkcolor=black,
    filecolor=magenta,      
    urlcolor=cyan,
}

\renewcommand{\baselinestretch}{1.5}

\title{SS004.6}
\author{Lê Hoàng Phúc, Vũ Thiên Phú}
\date{December 2021}

\begin{document}
\begin{titlepage}
\begin{center}
    \centering
    \LARGE
    \textbf {Đại học Quốc gia TP.HCM\\}
    \textbf {Trường Đại học Công nghệ Thông tin\\[1.5cm]}
\end{center}



\begin{figure} [!htp]
  \centering
  \includegraphics {Logo}
\end{figure}

\begin{center}
    \Large
    \textbf{\huge\\[0.01cm] Kỹ năng nghề nghiệp SS004.M13\\[0.5cm]}
    \textbf{Đề tài:\\}
    \textbf{\huge Đồ án cuối kỳ \\[0.5cm]}
    \textbf{Sinh viên thực hiện:\\[0.5cm]}
    \textbf{Trần Ngọc Tiến - 20520808\\}
    \textbf{Nguyễn Tấn Giang - 20521261\\}
    \textbf{Lê Hoàng Phúc - 20521763\\}
    \textbf{Vũ Thiên Phú - 20521758\\}
    \textbf{Nguyễn Hoàng Phúc - 20521768\\[1.5cm]}
    \textbf{Hồ Chí Minh, tháng 11 - 2021}
\end{center}
\end{titlepage}
\thispagestyle{empty}
\setcounter{page}{0}

\Large
\tableofcontents
\vspace{1 cm}
\thispagestyle{empty}
\setcounter{page}{0}

\pagebreak

\section {Hợp đồng nhóm 3PGT}
\vspace{0.25cm}
\subsection{Thành viên}
\vspace{1cm}
\begin{tabular}{ |c| c| }
\hline
 MSSV & Họ Tên Sinh Viên  \\ 
\hline
 20520808 & Trần Ngọc Tiến \\  
 \hline
 20521261 & Nguyễn Tấn Giang\\
\hline
 20521763 & Lê Hoàng Phúc \\  
 \hline
 20521758 & Vũ Thiên Phú\\
 \hline
  20521768 & Nguyễn Hoàng Phúc\\
\hline
\end{tabular}
\vspace{0.75cm}
\subsection{Mục tiêu nhóm}
\vspace{0.25cm}
\begin{itemize}
    \item Rèn luyện được kỹ năng làm việc nhóm
    \item Có được những kiến thức về các phần mềm như Latex, Trello,…  
    \item Nâng cao tinh thần đoàn kết, xây dựng mối quan hệ tốt đẹp giữa các thành viên
    \item 	Có tính kỷ luật cao, trung thực, tin cậy và có sự sáng tạo trong mỗi thành viên
    \item	Hoàn thành tốt đồ án đã giao đúng thời hạn
\end{itemize}

\subsection{Vai trò}
\vspace{1cm}
\begin{tabular}{ |c| >{\centering\arraybackslash}p{2 cm}|>{\centering\arraybackslash}p{2 cm}| >{\centering\arraybackslash}p{2 cm} | >{\centering\arraybackslash}p{2 cm} | }
\hline
\diagbox[]{Thành viên}{Vai trò}& Nhóm trưởng & Code Game & Báo cáo & Kiểm thử \\
\hline
 Trần Ngọc Tiến & x & x &  &\\  
 \hline
 Nguyễn Tấn Giang &  & x &  &\\  
 \hline
  Lê Hoàng Phúc &  &  & x & x\\  
 \hline
 Vũ Thiên Phú &  & x & x &\\  
 \hline
Nguyễn Hoàng Phúc &  & x &  &\\  
 \hline

\end{tabular}

\subsection{Chỉ tiêu đánh giá}
\vspace{1cm}
\begin{tabular}{ |>{\raggedright}p{3 cm}| >{\raggedright}p{3.25 cm}|>{\raggedright}p{2.75 cm}| >{\raggedright}p{3 cm} |>{\raggedright\arraybackslash}p{2.5 cm}| }
\hline
\diagbox[innerwidth=3cm]{Tiêu chí}{Mức độ} & Xuất sắc& Tốt &Trung bình &Kém\\
\hline
Thái độ làm việc&Hoàn thành tốt nhiệm vụ được giao, giúp đỡ các thành viên trong nhóm nhiệt tình&Hoàn thành nhiệm vụ được giao&Hoàn thành nhiệm vụ được giao nhưng vẫn còn có sự nhắc nhở&Không hoàn thành nhiệm vụ được giao\\
\hline
Quản lí thời gian&Hoàn thành nhiệm vụ trước thời hạn và đúng giờ trong những buổi họp nhóm&Hoàn thành nhiệm vụ đúng thời hạn và trễ không quá 5 phút trong các buổi họp nhóm& Hoàn thành nhiệm vụ trễ thời hạn đã giao(trọng hạn mức) và trễ 5-10 phút ở các buổi họp&Không hoàn thành nhiệm vụ và trễ quá 10 phút trong các buổi họp (hoặc không họp)\\
\hline
Đóng góp, nêu ý kiến&Sẵn sàng nêu ra những ý kiến , đóng góp tích cực &Có nêu ra ý kiến, có sự đóng góp&Có nêu ra ý kiến,đóng góp nhưng còn hạn chế cần sự nhắc nhở&Không đưa ra những ý kiến hay đóng góp gì cho nhóm\\
\hline
\end{tabular}
\begin{tabular}{ |>{\raggedright}p{3 cm}| >{\raggedright}p{3.25 cm}|>{\raggedright}p{2.75 cm}| >{\raggedright}p{3 cm} |>{\raggedright\arraybackslash}p{2.5 cm}| }
\hline
 Tìm kiếm , tổng hợp thông tin&	Tìm kiếm thông tin đầy đủ, đa dạng, phong phú, tổng hợp  hiệu quả&	Tìm kiếm thông tin đầy đủ, tổng hợp thông tin hiệu quả&	Tìm kiếm thông tin sơ sài, tổng hợp chưa quá hiệu quả&	Không tìm kiếm thông tin\\ 
 
 \hline
 Tinh thần hợp tác&	Sôi nổi, trách nhiệm, có quan hệ tốt với thành viên, đốc thúc mối quan hệ trong nhóm, giải quyết mâu thuẫn nội bộ&	Giữ được  quan hệ tốt đẹp với thành viên trong nhóm, giải quyết được những mâu thuẫn nội bộ & Có sự tôn trọng lẫn nhau trong nhóm dựa trên mối quan hệ hợp tác, không gây xích mích, chia rẽ nội bộ&	Không có sự tôn trọng với các thành viên trong nhóm, gây xích mích, chia rẽ nội bộ\\
 \hline

\end{tabular}
\pagebreak
\subsection{Quy định, hiệp ước}
\vspace{0.25cm}

\begin{itemize}
    \item Nhóm hoạt động theo mô hình tư vấn
    \item Ra quyết định dựa trên quyết định của nhóm trưởng (tham vấn các thành viên trước khi ra quyết định)
    \item Tôn trọng các thành viên trong nhóm
    \item Nói không với việc thiếu trung thực, copy bài những nhóm khác
    \item Trong lúc thảo luận, có người đưa ra ý kiến thì những người còn lại phải lắng nghe, sau khi người kia ý kiến xong thì mới đưa ra quan điểm của mình
    \item Hoàn thành công việc đúng thời hạn và chất lượng công việc phải được đảm bảo
\end{itemize}
\vspace{1 cm}
\newpage
\subsection{Ký tên} 

Chữ kí thành viên \\

\begin{tabular}{p{0.45\textwidth}cp{0.45\textwidth}}

  \includegraphics[width=\linewidth]{TNT.png} & & \includegraphics[width=\linewidth]{Giang.png} \\

  \centering \href { https://www.facebook.com/ngoctien.TNT } { Trần Ngọc Tiến } &  & \centering \href { https://www.facebook.com/SepGiang15th } { Nguyễn Tấn Giang }
\end{tabular} \\

\begin{tabular}{p{0.45\textwidth}cp{0.45\textwidth}}

  \includegraphics[width=\linewidth]{PhucLe.jpg} & & \includegraphics[width=\linewidth]{Phu.png} \\

  \centering \href{https://www.facebook.com/profile.php?id=100004563106812}{ Lê Hoàng Phúc } &  & \centering \href{https://www.facebook.com/profile.php?id=100003962510921}{ Vũ Thiên Phú }
\end{tabular} \\[2cm]

\begin{tabular}{p{0.45\textwidth}}

  \includegraphics[width=\linewidth]{PhucNguyen.png} \\

  \centering \href{https://www.facebook.com/nhphuc0703}{ Nguyễn Hoàng Phúc }
\end{tabular} \\
\newpage
\subsection{Không gian làm việc}
\begin{itemize}
   \item \href { https://github.com/ngoctienUIT/SnakeGame } { Github }
    \item \href { https://trello.com/b/aph53kLG } { Trello }
\end{itemize}
\newpage
\section{Phần giới thiệu và hướng dẫn chơi game }
\subsection{SnakeGame là gì?}
\setlength{\parindent}{0.5 cm}
SnakeGame là một game dựa trên game Rắn săn mồi truyền thống- một tựa game đình đám đã gắn liền với bao tuổi thơ của thanh niên Việt Nam nói riêng và trên khắp thế giới nói chung. Có thể nói Rắn săn mồi là phát súng đầu tiên cho thế hệ các trò chơi điện tử sau này, đặc biệt ngày nay có nhiều phiên bản biến tấu mới của nó đang nổi đình nổi đám như Slither.io. Nhưng dù vậy thì chế độ rắn săn mồi truyền thống vẫn có một sức hút to lớn với nhiều người bởi lối chơi đơn giản, có tính giải trí cao. Đến với game SnakeGame này bạn sẽ được trải nghiệm tựa game này với chế độ chơi ngày xưa để hoài niệm về những ký ức tuổi thơ đồng thời SnakeGame còn được nâng cấp lên  một phiên bản màu sắc và sinh động hơn rất nhiều so với phiên bản trắng đen trên những chiếc điện thoại ngày xưa. Trong game nhiệm vụ của bạn vẫn là điều khiển chú rắn di chuyển khéo léo để có thể lấy được thật nhiều các loại trái cây . SnakeGame  với hai chế độ chơi khác nhau đang chờ đón bạn trải nghiệm cảm giác được trải nghiệm .
\subsection{Hướng dẫn chơi game}

\begin{itemize}
    \item  SnakeGame vì là một game ngày xưa nên cấu hình rất nhẹ, bạn có thể tải về và chơi  một cách nhanh chóng.
    \item Hướng dẫn cài đặt và sử dụng chương trình:  Cách cài đặt rất đơn giản bạn chỉ cần tải mã nguồn chương trình về và mở file SnakeGame.exe để bắt đầu chơi game.
    \item Hướng dẫn về giao diện game:
      \begin{itemize}
    \item  Khi mở chương trình bạn sẽ thấy một cửa sổ mới được mở lên  trong đó có các mục như NewGame , HighScore , QuitGame, About :
    \begin{itemize}
    \item  NewGame để bắt đầu 1 game mới
	\item HighScore để xem bảng xếp hạng của 5 người chơi có điểm số cao nhất
	\item QuitGame để thoát khỏi chương trình.
	\item About là thông tin về chương trình.
    \end{itemize}
	
\item Bạn sử dụng con trỏ chuột và click vào mục bạn muốn chọn. Riêng đối với NewGame thì bọn sẽ chọn 1 trong 2 chế độ chơi Classic hoặc Modern và với mỗi chế độ chơi thì có 5 cấp độ, với độ khó tăng dần từ 1 đến 5. Với 2 mục About và HighScore thì để trở về menu chính bạn chỉ cần click chuột ở nơi bất kỳ trong cửa sổ.
\end{itemize}
\item Hướng dẫn cách điều khiển
\begin{itemize}
    \item Về cơ bản thì cách điều khiển rắn của SnakeGame hoàn toan giống với cách điều khiển trong quá khứ là sử dụng các phím mũi tên để chuyển hướng.
    \item Lưu ý là bạn chỉ có thể chuyển hướng từ phương ngang sang phương dọc và ngược lại.
    \item Lúc đang chơi nếu muốn tạm dừng bạn sẽ nhấn phím Space và nhấn phím ESC để kết thúc game ngay lập tức.
    \item Sau khi kết thức trò chơi bạn sẽ chờ một khoảng thời gian ngắn rồi nhập tên của minh và nhấn Enter (tên không được có dấu cách). Nếu điểm số của bạn nằm trong 5 người cao điểm nhất thì sẽ được lưu trữ lại.
\end{itemize}

\item Hướng dẫn về lối chơi
\begin{itemize}
     \item Về cơ bản thì cả hai chế độ Classic và Modern có cách chơi khá giống nhau, chỉ khác nhau ở một vài điểm nhỏ.
\item Với chế độ Classic:
\begin{itemize}
    \item Bạn sẽ hóa thân thành một chú rắn và sử dụng kỹ năng điều khiển một cách khéo léo để ăn được thức ăn.
	\item Sau mỗi lần ăn được, cơ thể của chú rắn sẽ tăng lên 1 đốt và điểm số của bạn tất nhiên cũng sẽ tăng lên
    \item Nếu Rắn đi ra ngoài mép tường – cho rắn xuất hiện lại ở phía bên kia.
    \item Bạn sẽ thua khi ăn phải thân minh 
\end{itemize}
	\item Với chế độ Modern:
    \begin{itemize}
    \item Về cơ bản thì lối chơi giống Classic bạn sẽ điều khiển chú rắn ăn thức ăn nhiều nhất có thể  nhưng sẽ tăng thêm độ khó nữa là bạn không thể đi xuyên tường, trò chơi sẽ kết thúc khi bạn va vào tường hoặc tự ăn chính bản thân.
\end{itemize}
    
\end{itemize}
\end{itemize}

\section{Tài liệu kỹ thuật}
\subsection{Nền tảng,công cụ phát triển:}
\begin{itemize}
    \item Để làm được SnakeGame với đồ họa hình ảnh và âm thanh thì bạn cần phải sử dụng các thư viện đồ họa có sẵn như : winbgim.h .
	\item Nhóm em đã chọn DevC++ mà không phải Visual Studio làm IDE để phát triển SnakeGame vì sự tiện dụng , cấu hình của DevC++ rất nhẹ có thể sử dụng trên nhiều máy tính thêm vào đó là việc cài đặt thư viện dễ dàng.
	\item Để có thể lập trình đồ họa bằng C++ trên DevC++ thì bạn cần phải cài đặt thư viện winbgim.h vào DevC++ theo các bước sau:
	\begin{itemize}
	    \item Tải DevC++ về máy(link:\href { https://sourceforge.net/projects/embarcadero-devcpp/?fbclid=IwAR28tY1k2jzEnAbQSJlN0Q3Jg741Yeg2CxNU95PquCyH-XCROjNtL2ZfUW8} {DevC++})
	\item Tải thư viện đồ họa về máy: 
(link:\href { https://drive.google.com/file/d/1iA5XMpK5XhypoUQ5ff2jejrrzEIvomM2/view?usp=sharing} { Graphics })
\item Giải nén file thư viện.
\item Copy file libbgi.a vào thư mục lib 
\item Copy file winbgim.h và graphics.h vào thư mục include .
\item Copy 2 file 6-ConsoleAppGraphics.template và file\\ ConsoleApp\_cpp\_graph.txt vào thư mục Templates  
	\end{itemize}
\end{itemize}
\subsection{Cấu trúc sử dụng cho các đối tượng:}
\begin{itemize}
    \item Tổng quát hóa các đối tượng
    \begin{itemize}
        \item Con rắn săn mồi của chúng ta sẽ là một chuỗi các hình tròn nhỏ (các đốt của con rắn) nối lại với nhau (số hình tròn nhỏ chính là độ dài của con rắn). Khởi tạo trò chơi, Ta đặt độ dài ban đầu của nó là 3. Trong quá trình trò chơi diễn ra, ta phải lưu vết được tọa độ và bổ sung thêm số lượng các hình tròn đó.
        \item  Tại mỗi bước dịch chuyển của rắn, mỗi đốt thân sẽ di chuyển đi 1 đơn vị độ dài bằng nhau. Ta chỉ cần nắm bắt đốt thân đầu tiên (đầu của rắn) tiến lên theo hướng di chuyển, các  đốt thân phía sau di chuyển đến vị trí cũ của đốt thân phía trước nó.
        \item Thức ăn của rắn được coi là 1 hình tròn tương tự như một đốt thân của rắn.Tại mỗi thời điểm vị trị của thức ăn là ngẫu nhiên.
    \end{itemize}

\item Chuyển đối tượng sang ngôn ngữ lập trình:
\begin{itemize}
    \item Một con rắn sẽ là 1 mảng các đối tượng Point tương ứng với các đốt của rắn. Tại mỗi đốt, cặp tọa độ (x,y) sẽ lưu vị trí đốt hiện tại và cặp tọa độ (x0,y0) sẽ lưu vị trí trước đó của đốt hiện tại để các đốt sau đó của con rắn có thể sử dụng.
    \item Mỗi đối tượng thức ăn sẽ là 1 đối tượng Point. Ta chỉ cần sử dụng cặp biến (x, y) để lưu 1 đối tượng thức ăn. Tại một thời điểm trên màn hình chỉ có 1 thức ăn, và thức ăn đó xuất hiện ở 1 vị trí ngẫu nhiên bất kỳ.
    \item Để rắn di chuyển được trên màn hình thì ta cần thêm một biến lưu hướng đi của nó. Ta tận dụng đối tượng Point để xác định hướng theo tọa độ (x,y). Ví dụ nếu rắn đang di chuyển theo hướng trái sang phải, như vậy đối tượng direction của ta sẽ là (10, 0) tức là tọa độ x sẽ tăng thêm 10 đơn vị và tọa độ y không đổi. Để thay đổi hướng đi  ta chỉ cần thay đổi giá trị (x, y) của đối tượng direction. Khoảng cách mỗi lần di chuyển được ta định nghĩa bởi hằng số DIRECTION. 
\end{itemize}

\item	Cấu trúc Point gồm 4 thành phần : (x, y) lưu vị trị hiện tại, (x0,y0) lưu vị trị trước đó nếu là 1 đốt của rắn.
\end{itemize}
\subsection{Hiển thị các đối tượng ra màn hình:}
\indent Trước hết ta sẽ viết các hàm khởi tạo giá trị và xuất hình ảnh Rắn, thức ăn , tường, tiêu đề ra màn hình thì ta cần viết các hàm sau :
\begin{itemize}

\item void initGame(): để khởi tạo giá trị cho rắn (số lượn, tọa độ), Thức ăn(tọa độ),hướng di chuyển ban đầu, hình ảnh background cho các đối tượng cố định. Đơn giản chỉ là gán các giá trị cho các đối tượng và hiện thị hình ảnh background bằng các hàm đồ họa có sẵn trong thư viện như: outtextxy, settextstyle, setcolor,…
\item void drawPoint (int x,int y,int radius) đơn giản chỉ là vẽ một hình tròn ra màn hình với tâm có tọa độ (x,y) và bán kinh radius được truyền vào.
\item void drawSnake(): đơn giản chỉ gọi lại hàm drawPoint nhiều lần đề vẽ các đốt của rắn vói tọa độ tâm được truyền vào là tọa độ các đốt của rắn.
\item void drawFood(): Tương tự như drawSnake cũng là gọi lại hàm drawPoint với tọa độ truyền vào là tọa độ của thức ăn.
\item void drawGame(): chỉ đơn giản là gọi lại 2 hàm drawSnake và drawFood.
\item void showText(int x,int y,char *str): sẽ hiện thị ra màn hình chuỗi str tại vị trị (x,y) với màu sắc thay đổi liên tục sau một khoảng thời gian(sử dụng hàm delay).
\item void showTextBackground(int x,int y,char *str,int color) tương tự như hàm showText nhưng có thêm màu sắc của background xung quanh là cố định được truyền vào qua tham số color.

\end{itemize}
\subsection{Viết các hàm xây dựng tinh logic game.}
\indent Ở đây ta sẽ tiến hành xây dựng logic cho game như: di chuyển, GameOver,...
\begin{itemize}
    \item void changeDirecton (int x): để thay đổi hướng đi ta chỉ cần thay đổi các giá trị (x,y) của biến direction dựa vào x và hướng di chuyển hiện tại. Với x ASCII của phím được nhập bởi bản phím ( x=72 đi lên ; x=80 hướng xuống; x=77 sang phải ; x=75 sang trái). Giá trị (x,y) của direction chỉ thay đổi khi có sự thay đổi về phương di chuyển nghĩa là : đang di chuyển theo phương ngang sau đó chuyển sang phương dọc và ngược lại.
    \item void classic(): chỉ đơn giản thay đổi tọa độ các đốt của rắn .Nếu là đốt thân rắn thì tọa độ mới bằng tọa độ cũ điểm đốt đằng trước nó. Nếu là đầu rắn thì chỉ đơn giản là cộng tọa độ đốt đầu rắn với tọa độ của hướng di chuyển direction. Nếu đầu rắn đụng tường thì ta không cộng với tọa độ direction mà thay bằng tọa độ của bức tường đối diện. Nếu đầu rắn trùng với thức ăn thì (tức là rắn đã ăn được thức ăn) thì ta tăng độ dài cho rắn và thay đổi vị trí mới cho thức ăn.
    \item void modern(): hàm này tương như hàm Classic nhưng khi đầu rắn đụng tường thay vì đi xuyên qua thì ta sẽ thêm điều kiện này là kết thúc trò chơi.
    \item void mainLoop (void (*xxx)()) : đây là vòng lập chính cho 1 ván game  tham số truyền vào của hàm này là 1 trong 2 hàm Classic hoặc Modern ở trên, và hàm này sẽ bắt  mã ASCII các phím được nhập từ bàn phím ở bằng hàm kbnit () và getch() được cung cấp sẵn để thay đổi hướng (direction) bằng cách truyền tham số vào hàm changeDirection(int x),  tạm dừng hoăc kết thúc game.
    \item bool checkPoint () : để đảm bảo vị trí mới của thức ăn không trùng với vị trí của rắn.
\end{itemize}
\subsection{Xây dựng các tiện ích cho chương trình:}
\indent Ở các bước trên ta đã hoàn thành cách điều khiển, di chuyển rắn, logic của game. Bây giờ ta sẽ viết thêm các tính năng như lưu điểm, menu , UI,…
\begin{itemize}

    \item Xây dụng các tiện ích:
    \begin{itemize}
        \item Để lưu lại điểm của người chơi ta sẽ tạo cấu trúc highScore gồm 2 thành phần: name là tên người chơi , score là điểm của người chơi đó.
	    \item Ta cần tạo 1 file highscore.txt để lưu lại điểm số.
	    \item showHighScore () : Ta sẽ hiển thị ra màn hình thông tin 5 người chơi có số điểm cao nhất được  được lưu vào file highscore.txt .Nếu file highscore.txt rỗng thì mặc định tên của 5 người chơi đều bằng player và score=0.
	    \item bool isEmpty() hàm này sẽ giúp ta kiểm tra xem file highscore.txt có rỗng không.
	    \item void initScore() sẽ gọi lại hàm isEmpty xem file highscore.txt có rỗng rỗng không bằng cách sử dụng hàm isEmpty().Nếu rông thì khởi tạo tên của 5 người chơi đều là Player và score=0.
	    \item void getHighScore () : cập nhật lại danh sách 5 thành viên điểm cao vào file highscore.txt .
	    \item void checkHighScore(int score) hàm này sẽ giúp bạn nhập tên người chơi sau khi kết thúc kiểm tra xem số điểm của bạn đạt được có thuộc top 5 người đứng đầu không. Nếu có thì sẽ gọi hàm getHighScore để lưu dữ liệu của bạn vào file highscore.txt. 
    \end{itemize}
	
    \item Xây dựng UI:
    \begin{itemize}
        \item Để tạo thiện cảm với người dùng lúc mở chương trình ta sẽ tạo một màn hình tải game từ 0% đến 100% qua hàm void loadingScreen(). 
        \item Hàm run():
        \begin{itemize}
            \item  Đây là hàm chạy cho toàn bộ chương trinh, nó sẽ sử dụng các hàm đã viết ở trên.
            \item Đầu tiên sẽ gọi hàm loadingScreen để tạo hiệu ứng tải load game. 
            \item Sủ dụng các hàm showTextBackground, setbkcolor, ShowText , … để hiển thị ra màn hình tại các vị trí cụ thể. Sử dụng ismouseclick bắt bắt tình huống click chuột, mousex là tọa độ x của con trỏ chuột, mousey là tọa đọ trục Oy của  con trỏ chuột . Kết hợp với các lệnh điều kiện ràng buộc vị trí click chuột để gọi các hàm.
            \item Để truyền âm thanh vào ta dùng hàm PlaySound(TEXT(“x”)) với x là đường dẫn vào file âm thanh.
        \end{itemize}

    \end{itemize}
 
\end{itemize}
\section{Mô tả quá trình làm việc nhóm }
\subsection{Quá trình thành lập nhóm }

Ngay sau khi được thông báo làm bài tập đồ án cuối kỳ dựa trên làm
việc nhóm với tối thiểu 3 thành viên và tối đa 5 thành viên, mọi người đều ráo riết đi kiếm những mảnh ghép còn thiếu để hoàn thành việc thành lập nhóm. Bạn Lê Hoàng Phúc và bạn Nguyễn Tấn Giang là hai người đã liên lạc với nhau và đi đến quyết định gộp chung 2 nhóm 2 người (2 nhóm được thành lập trong quá trình làm bài tập SS004.M13.6). Không chỉ dừng lại ở đó, nhóm chúng em đã quyết định thêm 1 thành viên mới là bạn Nguyễn Hoàng Phúc, hoàn thiện bộ khung 5 người, sẵn sàng để cùng nhau hoàn thành đồ án cuối kì này một cách tốt nhất và chất lượng nhất.
\subsection{Làm hợp đồng nhóm }
\begin{itemize}
    \item Sau khi được nhận bài tập đầu tiên là làm hợp đồng nhóm, cả nhóm quyết định chọn 1 ngày để bàn về việc làm hợp đồng, đồng thời cũng làm quen với nhau để tiện cho việc trao đổi và hỗ trợ nhau làm bài tốt hơn.
    \item 19h ngày 22/11/2021, nhóm quyết định họp ở Google Meet, tất cả đều tham gia đông đủ và đúng giờ
    \item Nội dung cuộc họp:
        \begin{itemize}
            \item Giới thiệu bản thân cho từng thành viên còn lại.
            \item Bầu trưởng nhóm:  Cả nhóm thống nhất với việc bầu bạn Trần Ngọc Tiến làm trưởng nhóm (100\% phiếu bầu)
            \item Đặt tên nhóm: Sau một hồi đưa ra những ý kiến về tên nhóm khác nhau, cả nhóm đã thống nhất tên của nhóm sẽ là 3PGT – tương ứng với những chữ cái đầu của tên mỗi bạn (3P ở đây là 3 cái chữ P tương ứng với 3 bạn Vũ Thiên Phú, Lê Hoàng Phúc và Nguyễn Hoàng Phúc, G tương ứng với bạn Nguyễn Tấn Giang, T tương ứng với bạn Trần Ngọc Tiến).
            \item Nhóm quyết định lấy 2 mẫu hợp đồng nhóm trước đó làm ở bài tập SS004.M13.6 tham khảo và chỉnh sửa lại thành phiên bản hợp đồng nhóm hoàn chỉnh sao cho phù hợp với đồ án cuối kỳ.  Việc soạn thảo nội dung và gõ Latex sẽ do bạn Tiến đảm nhiệm chính, các bạn còn lại trong nhóm sẽ đưa ra những ý kiến và hỗ trợ bạn Tiến trong việc làm hợp đồng nhóm này.
            \item Ở trong buổi họp này, nhóm trưởng là bạn Tiến đã tạo Trello và GitHub và add các thành viên trong nhóm và sau đó đã add thầy vào để dễ dàng kiểm soát quá trình làm việc của nhóm. Nhóm trưởng cũng đã khuyến khích các bạn tìm hiểu về GitHub, tạo tài khoản GitHub với tên theo yêu cầu của bài tập tuần 2 – UIT.MSSV, tìm hiểu cách tạo branch, push code, pull request,… để đến khi làm không bị bỡ ngỡ.
            \item Sau cùng, nhóm hẹn đến ngày 28/11/2021 sẽ họp 1 lần nữa trên Google Meet để chỉnh sửa và hoàn thiện hợp đồng nhóm.
        \end{itemize}
    \item 16h, ngày 28/11/2021, cuộc họp thứ 2 diễn ra
    \item Nội dung cuộc họp:
        \begin{itemize}
            \item Bạn Tiến đã hoàn thành gần xong hợp đồng nhóm, các thành viên đưa ra những lời nhận xét và góp ý.
            \item Thảo luận và phân công các thành viên trong nhóm làm những công việc cụ thể trong mục Vai trò của hợp đồng nhóm.
            \item Chỉnh sửa lại những lỗi trong khi soạn thảo Latex
        \end{itemize}
    \item Ngày 30/11/2021, nhóm trưởng nộp bài tập đồ án cuối kỳ tuần 1 trên course.
    \item Sau đó, nhóm trưởng đã up mã nguồn cơ bản (mức độ như code mà thầy đã đưa lên courses) lên GitHub, hoàn thiện lại hợp đồng nhóm 1 lần nữa và gắn kèm link Trello, GitHub và nộp bài tập đồ án cuối kỳ tuần 2 lên courses.
\end{itemize}
\subsection{Code game và làm báo cáo}
\begin{itemize}
    \item Sau khi hoàn thành xong hợp đồng nhóm, các thành viên trong nhóm tích cực tìm những source code trên mạng cho bài tập SnakeGame này. 
    \item 19h, ngày 5/12/2021, nhóm quyết định họp lần thứ 3 ở Google Meet để giải quyết bài tập đồ án cuối kỳ.
    \item Nội dung cuộc họp:
    \begin{itemize}
        \item Nhóm trưởng đưa các source code , và yêu cầu mọi người tham khảo và chọn một IDE để code và   cải tiến lại để làm bài tập SnakeGame.
        \item Nhóm trưởng phân công các bạn code chia nhỏ từng công việc, mỗi bạn sẽ đảm nhận một vài phần khác nhau,  cải tiến lại và push code lên Git. Cụ thể:
            \begin{itemize}
                \item  Bạn Giang và bạn Phú sẽ đảm nhận việc Xây dựng cấu trúc cho Point, các hàm hiển thị ra màn hình. (deadline 6/12-7/12).
                \item Bạn Tiến và bạn Nguyễn Hoàng Phúc sẽ đảm nhận việc Xây dựng logic của game.(deadline 7/12 – 9/12).
                \item Bạn Tiến, bạn Giang và bạn Nguyễn Hoàng Phúc sẽ đảm nhận việc Xây dựng thêm tiện ích cho game (UI).(deadline 9/12-11/12).
            \end{itemize}
        \item Và cuối cùng, nhóm trưởng giao cho bạn Lê Hoàng Phúc nhiệm vụ test game sau khi đã push hết code lên Git (deadline 11/12-12/12).
        \item Nhóm trưởng khuyến khích tinh thần sáng tạo của các thành viên trong nhóm, cũng như khuyến khích các thành viên trong nhóm hỗ trợ nhau trong quá trình làm bài để có thể hoàn thành bài tập đúng hạn với chất lượng tốt nhất.
        \item Song song với việc code, nhóm trưởng còn dặn dò hai bạn Phú và Lê Hoàng Phúc tiến hành làm bài báo cáo để có thể kịp tiến độ. (deadline báo cáo 7/12 – 14/12).
        \end{itemize}
    \item Cả nhóm đã bắt tay nhau làm việc sau khi buổi họp kết thúc, push code lên đúng thời hạn, thậm chí là sớm hơn thời hạn đã giao. Tinh thần hăng hái làm việc của nhóm cùng với sự giúp đỡ, hỗ trợ nhau mỗi khi có thành viên trong nhóm cần đã giúp nhóm làm việc với công suất tối đa, hoàn thành SnakeGame rất chất lượng.
    \item Ngày 14/12/2021, nhóm ngồi lại họp với nhau một lần nữa để dò xét lại báo cáo và code thật kĩ càng, cùng với đó nhóm trưởng sẽ đánh giá phần làm việc nhóm của từng bạn và các thành viên còn lại sẽ đánh giá nhóm trưởng.
\end{itemize}
\subsection{Những thuận lợi và khó khăn trong khi làm việc nhóm}
\subsubsection{Thuận lợi}
\begin{itemize}
    \item Cả 5 thành viên trong nhóm đều tuân thủ nghiêm ngặt quy định nhóm đã giao, hoàn thành deadline đúng thời hạn.
    \item Nhóm trưởng có tố chất lãnh đạo, gắn kết các thành viên trong nhóm với nhau thành 1 khối thống nhất.
    \item Các thành viên còn lại đều năng nổ, tích cực đóng góp hết sức mình vào bài tập đồ án cuối kỳ.
    \item Không xảy ra quá nhiều tranh cãi giữa các thành viên, nếu có cũng chỉ là những ý kiến trái chiều từ vài bạn và đều đã được giải quyết tốt đẹp.
    \item Các thành viên trong nhóm giúp đỡ tương trợ nhau hết mình để hoàn thành deadline kịp hoặc sớm hơn thời hạn đã giao.
\end{itemize}
\subsubsection{Khó khăn}
\begin{itemize}
    \item Ban đầu,nhóm không có sự thống nhất chung giữa các ý kiến. Có bạn đã đề ra ý tưởng làm game khác, có bạn đề ra ý tưởng code Snake Game bằng ngôn ngữ Python nhưng sau khi biểu quyết những lựa chọn trên thì số đông vẫn ủng hộ rằng code Snake Game bằng ngôn ngữ C++ như bài tập yêu cầu.
    \item Khi tham khảo source code của nhóm trưởng đưa thì đã có những lỗi xảy ra không như mong muốn, xuất phát từ nhiều bạn, nhưng sau đó nhóm đã tìm cách giải quyết những lỗi đó và quyết định dùng source code kia.
    \item Có một vấn đề đã xảy ra với nhóm. Một bạn trong nhóm đã vô tình up toàn bộ code lên GitHub nên sau đó nhóm đã phải xóa Repo đi và tạo lại một cái mới hoàn toàn.
\end{itemize}

\section{Các kỹ năng nhóm đã áp dụng trong đồ án cuối kỳ}
Nhóm đã áp dụng rất nhiều kỹ năng trong bài tập nhóm đồ án cuối kỳ lần này
\begin{itemize}
    \item Giao tiếp: Nhóm đã giao tiếp rất nhiều, từ Messenger đến qua Google Meet. Việc giao tiếp giúp nhóm tăng tính tương tác giữa các thành viên, giúp các thành viên trong nhóm hiểu nhau hơn, hạn chế tối đa những mâu thuẫn khi làm việc nhóm
    \item Giải quyết vấn đề: Các thành viên trong nhóm đều trang bị cho bản thân khả năng problem solving khá tốt. Mọi người luôn bình tĩnh giải quyết những vấn đề nảy sinh trong quá trình làm bài, hạn chế những hậu quả của vấn đề ở mức thấp nhất có thể hoặc triệt tiêu vấn đề. 
    \item Lắng nghe: Người nói thì phải có người nghe.Lắng nghe giúp các thành viên trong nhóm tiếp nhận thông tin một cách cởi mở hơn, giúp nhìn nhận ra những vấn đề theo nhiều khía cạnh.
    \item Tư duy phản biện: Không phải lúc nào 1 ý kiến đưa ra cũng là tối ưu nhất. Nhóm luôn đề cao những thành viên có những cách tư duy, nhìn nhận vấn đề theo chiều hướng mới mẻ, sáng tạo nhưng hiệu quả, sẵn sàng đưa ra những ý kiến của cá nhân mình để nhóm xem xét giải quyết theo chiều hướng tối ưu nhất.
    \item Quản lý thời gian: Yếu tố thời gian là một yếu tố quan trọng trong làm việc nhóm.Hoàn thành deadline trước thời hạn, thậm chí là trước thời hạn luôn được nhóm ưu tiên. Những cuộc họp hay những cuộc trao đổi luôn cần có sự hiện diện của tất cả mọi người để có thể cùng nhau xây dựng đồ án một cách tốt nhất, hiệu quả nhất
    \item Tôn trọng: Mỗi thành viên trong nhóm đều thể hiện sự tôn trọng dành cho những thành viên còn lại, không gây xích mích, tranh cãi gay gắt hay chia bè kéo phái làm rạn nứt sự đoàn kết giữa các thành viên trong nhóm.
    \item Tinh thần hợp tác, giúp đỡ lẫn nhau: Các thành viên trong nhóm đã ra sức giúp đỡ lẫn nhau trong xuyên suốt quá trình làm đồ án cuối kỳ. Giúp đỡ lẫn nhau giúp cho các thành viên trong nhóm bổ trợ cho nhau những mặt thiếu sót, tăng tình đoàn kết và có được những người bạn tốt đi với nhau suốt chặng đường dài Đại học này.
    \item Trách nhiệm và kỷ luật: Các thành viên trong nhóm luôn tuân thủ quy tắc mà nhóm đã thống nhất đưa ra. Không ai bỏ deadline, không ai bỏ họp nhóm, không ai thờ ơ với việc làm bài tập nhóm, không ai ỷ lại sự giúp đỡ của người khác. Các thành viên luôn xây dựng và hoạt động hết năng suất của mình.
\end{itemize}
\section{Đánh giá  việc thực hiện hợp đồng nhóm}
\begin{itemize}
    \item \indent Nhóm trưởng sẽ đánh giá các thành viên trong nhóm và các thành viên sẽ đánh giá nhóm trưởng.\\
\end{itemize}
\newpage
\begin{itemize}
    \item Nhóm trưởng đánh giá các thành viên.\\
    \begin{tabular}{ |>{\centering\arraybackslash}p{3 cm}| >{\centering\arraybackslash}p{3 cm}|>{\centering\arraybackslash}p{2 cm}| >{\centering\arraybackslash}p{1 cm} |>{\centering\arraybackslash}p{2 cm}|>{\centering\arraybackslash}p{1 cm}| }
\hline
Thành viên& \diagbox[innerwidth=3cm]{Tiêu chí}{Mức độ}&Xuất sắc&Tốt&Trung bình&Kém\\
\hline
\multirow{3}{3 cm} {Vũ Thiên Phú}& Thái độ& x&& &\\
\cline{2-6}
& Quản lý thời gian &x & & & \\
\cline{2-6}
& Đóng góp nêu, ý kiến &x& & & \\
\cline{2-6}
& Tìm kiếm, tổng hợp thông tin &x& & & \\
\cline{2-6}
& Tinh thần hợp tác &x& & & \\
\cline{2-6}
&\multicolumn{5 }{l|}{Nhận xét: Kỹ năng làm việc nhóm tốt, năng nổ,hoàn }\\

&\multicolumn{5 }{l|}{thành xuất sắc công việc được giao, trước hạn.(10/10) }\\
\hline
\multirow{3}{3 cm} {Lê Hoàng Phúc}& Thái độ& x&& &\\
\cline{2-6}
& Quản lý thời gian &x& & & \\
\cline{2-6}
& Đóng góp, nêu ý kiến &x& & & \\
\cline{2-6}
& Tìm kiếm, tổng hợp thông tin &x& & & \\
\cline{2-6}
& Tinh thần hợp tác &x& & & \\
\cline{2-6}
&\multicolumn{5 }{l|}{Nhận xét: Kỹ năng làm việc nhóm tốt, năng nổ,hoàn }\\

&\multicolumn{5 }{l|}{thành xuất sắc công việc được giao, trước hạn.(10/10) }\\

\hline
\end{tabular}
\newpage
\begin{tabular}{ |>{\centering\arraybackslash}p{3 cm}| >{\centering\arraybackslash}p{3 cm}|>{\centering\arraybackslash}p{2 cm}| >{\centering\arraybackslash}p{1 cm} |>{\centering\arraybackslash}p{2 cm}|>{\centering\arraybackslash}p{1 cm}| }
\hline
Thành viên& \diagbox[innerwidth=3cm]{Tiêu chí}{Mức độ}&Xuất sắc&Tốt&Trung bình&Kém\\
\hline
\multirow{3}{3 cm} {Nguyễn Hoàng Phúc}& Thái độ&x && &\\
\cline{2-6}
& Quản lý thời gian &x & & & \\
\cline{2-6}
& Đóng góp, nêu ý kiến &x& & & \\
\cline{2-6}
& Tìm kiếm, tổng hợp thông tin &x& & & \\
\cline{2-6}
& Tinh thần hợp tác &x& & & \\
\cline{2-6}
&\multicolumn{5 }{l|}{Nhận xét: Kỹ năng làm việc nhóm tốt, năng nổ,hoàn }\\

&\multicolumn{5 }{l|}{thành xuất sắc công việc được giao, trước hạn.(10/10) }\\
\hline
\multirow{3}{3 cm} {Nguyễn Tấn Giang}& Thái độ& x&& &\\
\cline{2-6}
& Quản lý thời gian &x& & & \\
\cline{2-6}
& Đóng góp, nêu ý kiến &x& & & \\
\cline{2-6}
& Tìm kiếm, tổng hợp thông tin &x& & & \\
\cline{2-6}
& Tinh thần hợp tác &x& & & \\
\cline{2-6}
&\multicolumn{5 }{l|}{Nhận xét: Kỹ năng làm việc nhóm tốt, năng nổ,hoàn }\\

&\multicolumn{5 }{l|}{thành xuất sắc công việc được giao, trước hạn.(10/10) }\\

\hline
\end{tabular}
\newpage
\item Đánh giá nhóm trưởng.\\
\begin{tabular}{ |>{\centering\arraybackslash}p{3 cm}| >{\centering\arraybackslash}p{3 cm}|>{\centering\arraybackslash}p{2 cm}| >{\centering\arraybackslash}p{1 cm} |>{\centering\arraybackslash}p{2 cm}|>{\centering\arraybackslash}p{1 cm}| }
\hline
Nhóm trưởng& \diagbox[innerwidth=3cm]{Tiêu chí}{Mức độ}&Xuất sắc&Tốt&Trung bình&Kém\\
\hline
\multirow{3}{3 cm} {Trần Ngọc Tiến}& Thái độ& x&  & & \\
\cline{2-6}
& Quản lý thời gian &x & & & \\
\cline{2-6}
& Đóng góp, nêu ý kiến &x& & & \\
\cline{2-6}
& Tìm kiếm, tổng hợp thông tin &x& & & \\
\cline{2-6}
& Tinh thần hợp tác &x& & & \\
\cline{2-6}
&\multicolumn{5 }{l|}{Nhận xét của các thành viên: Kỹ năng lãnh đạo nhóm}\\

&\multicolumn{5 }{l|}{  tốt, năng nổ ,nhiệt tình,thân thiện, hoàn thành tốt  }\\
&\multicolumn{5 }{l|}{  công việc được giao . (10/10) }\\

\hline
\end{tabular}
\end{itemize}
\newpage
\begin{thebibliography}{}
\bibitem[1]{1}Cài đặt Graphic.h cho DevC++ \url{https://www.slideshare.net/tiktiktc/graphics-in-dev-c} 
\bibitem[2]{2}Cách sử dụng các hàm đồ họa  \url{https://www.programmingsimplified.com/c/graphics.h} 
\bibitem[3]{3}Tham khảo code bằng Python \url{https://quantrimang.com/huong-dan-code-game-ran-san-moi-bang-python-174196} 
\bibitem[4]{4}Tham khảo code bằng C++ \url{https://nguyenvanhieu.vn/huong-dan-code-game-ran-san-moi-trong-c/} 



\end{thebibliography}


\end{document}